%-*- latex; coding: iso-8859-1; buffer-file-coding-system: iso-8859-1 -*-
% Far�: Latin1 encoding
\documentclass[natbib]{sigplanconf}
%\documentclass[preprint,natbib]{sigplanconf}

\usepackage[latin1]{inputenc}
\usepackage[english]{babel}
%\usepackage{ae}
%\usepackage{aeguill}
%\usepackage{shortvrb}
%\usepackage{tabularx}
%\usepackage{longtable}
%\setlength{\extrarowheight}{2pt}
%\usepackage{amsmath}
%\usepackage{graphicx}
%\usepackage{color}
%\usepackage{multirow}
%\usepackage{ifthen}
%\usepackage[DIV12]{typearea}
%\usepackage[colorlinks=true,linkcolor=blue,urlcolor=blue]{hyperref}
\usepackage[colorlinks=false]{hyperref}

\newlength{\locallinewidth}

\conferenceinfo{ILC '09}{2009-03-22, Cambridge, MA.}
\copyrightyear{2009}
\copyrightdata{[to be supplied]}

% The following lines are ignored unless 'preprint' option specified.
\titlebanner{Preprint of an ILC'09 paper}
\preprintfooter{This is a preprint}

\title{XCVB: Improving Modularity for Common Lisp}
%\subtitle{Subtitle Text, if any}
\authorinfo{Fran�ois-Ren� Rideau}
           {ITA Software, Inc.}
           {fare@tunes.org}
\authorinfo{Spencer Brody}
           {Brown University}
           {sbrody88@gmail.com}

\hypersetup{
pdftitle={XCVB: Improving Modularity for Common Lisp},
pdfauthor={Francois-Rene Rideau {\textless}fare@tunes.org{\textgreater} and Spencer Brody {\textless}sbrody88@gmail.com{\textgreater}}
}
\raggedbottom

\begin{document}
\toappear{}
\maketitle
%\thanks{This presentation describes work done at ITA Software, Inc.}

\begin{abstract}
We present XCVB, a new open source system to build Common Lisp software.
XCVB notably features separate compilation.
This departs from the Common Lisp tradition
of compiling within a live world.
But separate compilation scales to very large systems,
and opens many possibilities for the future.
\end{abstract}

%\category{CR-number}{subcategory}{third-level}

%\terms
%term1, term2

%\keywords
%keyword1, keyword2

\setlength{\locallinewidth}{\linewidth}

\section{Introduction}

Despite plenty of advanced module systems
having been implemented in various Lisp dialects
(and languages from other families)
in the last 25 years,
the current state of the art for the Common Lisp community
hasn't improved much since 1980:
a central file defines the build graph,
and components are compiled and loaded
as successive side effects to the current Lisp world.

Our insight is to break the classic Lisp assumption of programming
in a single concrete Lisp world that is side-effected
as source code is sequentially compiled then loaded.
Instead, we propose a pure-functional approach of
building encapsulated components,
each compiled in its own isolated virtual Lisp world.

The traditional Lisp model may have been
ahead of its time in the 1970s and valid into the 1980s,
when memory and virtualization were expensive.
But it doesn't scale well to programming in the large in the 2000s,
with projects of tens of programmers with hundreds of files,
where the side-effects of processing various files
may interact in unexpected ways.
Our model, which is traditional for many other languages (like OCaml),
does scale to large projects and allows a concurrent and distributed build.


\section{Past Common Lisp build systems}

In the original DEFSYSTEM \cite{CHINE-NUAL} and its descendents,
a ``system'' is defined in a central file as
an acyclic graph of components
to be compiled and loaded into the current Lisp world,
arcs being dependencies.

The original DEFSYSTEM and
its once-popular portable descendent MK:DEFSYSTEM (from 1990 on)
required one to manually list all transitive dependencies of a component
in a statement separate from its definition
(see figure~\ref{dep:defsystem}).
The current state of the art and most popular descendent of DEFSYSTEM,
ASDF \cite{ASDF-Manual},
simplifies things tremendously
by automatically computing the transitive closure
of directly declared dependencies
from single statements defining both both component and direct dependencies
(see figure~\ref{dep:asdf}).
Such improvements and more had been suggested as early as 1984
\cite{Pitman-Large-Systems,AITR-874}
but not available as a portable package until ASDF was created in 2001.
ASDF also features some limited form of extensibility using CLOS;
but for the purpose of this presentation,
we will focus on the core architecture of the build system.
Another DEFSYSTEM variant, mudballs, was recently created,
that seems to cleanup and simplify ASDF
while keeping the same general architecture.

\begin{figure}[tbp]
\begin{verbatim}
  (:module BAR ("BAR"))
  ...
  (:compile-load BAR
     (PACKAGES MACROS SPECIALS)
     (:fasload PACKAGES MACROS SPECIALS))
\end{verbatim}
\caption{Using DEFSYSTEM:
        component {\tt bar} is defined in two places in system file {\tt foo.system}
        -- all transitive dependencies have to be listed explicitly.}
\label{dep:defsystem}
\end{figure}

\begin{figure}[tbp]
\begin{verbatim}
  (:file "bar" :depends-on ("macros" "specials"))
\end{verbatim}
\caption{Using ASDF:
        component {\tt bar} is defined somewhere in system file {\tt foo.asd}
        -- transitive dependency on {\tt packages} may be skipped.}
\label{dep:asdf}
\end{figure}

\begin{figure}[tbp]
\begin{verbatim}
#+xcvb
(module
 (:depends-on ("macros" "specials")))
\end{verbatim}
\caption{Using XCVB:
        component {\tt bar} is defined on top of module file {\tt bar.lisp}
        -- transitive dependency on {\tt packages} is implicit.}
\label{dep:xcvb}
\end{figure}

\section{XCVB}

XCVB, the eXtensible Component Verifier and Builder,
is a new open source system to build software written in Common Lisp.

XCVB most notably features separate compilation in isolated Lisp worlds.
The current implementation spawns a new Lisp process for each compilation.
In XCVB, dependency information is not declared in a central file,
but at the top of each component file
(see figure~\ref{dep:xcvb}).

In the following bullet-point presentation,
we briefly outline the reasons why separate compilation is important,
though it requires a break from the traditional Common Lisp model.
We describe the state of the project and the benefits at hand.
We suggest the many future benefits that can be enabled by this design,
and we discuss the challenges that will have to be overcome
to achieve some of those benefits.

%___________________________________________________________________________

\section{XCVB yesterday}

What the XCVB prototype already brings as compared to ASDF.

\subsection{Goal: Separate Compilation}

\begin{itemize}
\item
Independent compilation of individual files.

\item
Compute objects from source, just as in any modern language.

\item
Proper staging of compile-time dependencies (macros compiled before they are used).

\item
Semantics of a file fully encapsulated in its contents (+ contents of dependencies).

\item
Incremental change-driven building and testing.

\end{itemize}

\subsection{Therefore: Dependencies must be declared locally}
\begin{itemize}

\item
Move dependencies away from centralized off-file meta-data.

\item
Module import statement, just as in any modern language.

\item
No more global recompilation (or subtle failure)
at the least change in centralized system definition.

\item
No more subtle bugs due to non-local change
in ordering of compile-time side-effects
(see figure \ref{asdf-break-down}).

\item
Unlike ASDF, can incrementally track dependencies across systems.

\end{itemize}

\begin{figure}[tbp]
Step 1: Initially working system.
\begin{verbatim}
  ;; bar provides base for baz
  (:file "bar" :depends-on ("base"))

  ;; quux unrelatedly depends on base
  (:file "quux" :depends-on ("base"))

  ;; baz also depends on base,
  ;; transitively provided by bar
  (:file "baz" :depends-on ("bar"))
\end{verbatim}

\vskip 1em

Step 2: after refactoring, {\tt bar} is simplified to not need {\tt base}
\begin{verbatim}
  ;; bar no longer depends on base
  (:file "bar" :depends-on ("packages"))

  ;; quux happens to load base
  ;; before baz is compiled
  (:file "quux" :depends-on ("base"))

  ;; baz really depends on base
  ;; but the missing dependency is not detected
  (:file "baz" :depends-on ("bar"))
\end{verbatim}

\vskip 1em

Step 3: {\tt quux} is also refactored to not need {\tt base}.
Now compilation of unrelated file {\tt baz} breaks
because its {\em implicit} dependency {\tt base} is not loaded anymore.
\begin{verbatim}
  ;; bar still does not depend on base
  (:file "bar" :depends-on ("packages"))

  ;; quux no longer loads base
  (:file "quux" :depends-on ("packages"))

  ;; despite a lack of related modifications,
  ;; baz now breaks inexplicably
  ;; at next clean compilation
  (:file "baz" :depends-on ("bar"))
\end{verbatim}

The lines that matter need not be consecutive
but may be separated arbitrarily.
The missing dependency may not be a direct dependency,
but any transitive dependency.
Steps 2 and 3 may be separated by a lot of unrelated changes.
Also, they need not be done by the same person,
rewarding the culprit, punishing the innocent,
encouraging sloppiness and discouraging refactoring.

%First workaround: always build from clean
%(and lose any incremental building advantage).
%Second workaround: use a rigid serial dependency order
%(and lose most incremental building, accumulate coding scars).

%With XCVB, when you modify {\tt bar}, {\tt baz} automatically gets recompiled,
%and any missing dependency is immediately detected.
%No timebomb. No need for a workaround.

\caption{Break down of ASDF:
	a mistake at step 2 creates a timebomb
        that is only triggered
        by an innocent change much later.}
\label{asdf-break-down}
\end{figure}


\subsection{Eager enforcement of dependencies}
\begin{itemize}

\item
Build each file in a world loaded with none but its dependencies.

\item
Requires import discipline, just as in any modern language.

\item
A bit slower, but much more robust:
dependency bugs are detected early
(as opposed to figure \ref{asdf-break-down}).

\item
No more unmaintainable large manual dependency graphs
(or rigid unmaintained serial lists of files as a workaround).

\item
Allows correct incremental unit tests based on what has changed
(beware: tests that use reflection).

\end{itemize}

\subsection{Current build backends}
\begin{itemize}

\item
XCVB computes the build graph, currently lets other software do the build.

\item
{\tt Makefile}: integrate into a larger build, just as in any modern language.

\item
ASDF: integrate into legacy ASDF builds.

\item
More backends possible in the future:
{\tt OMake} \cite{OMake}; take over your build.

\end{itemize}

\subsection{Decoupling builder and buildee}
\begin{itemize}
\item
Protection from uncontrolled side-effects from buildee to builder.

\item
Allows for integration with {\tt make} as mentioned above.

\item
Allows for cross-compilation from one compiler/architecture to a different one.

\item
Allows for a feature-rich build system that needn't fit in one small file, yet.

\item
Allows builder to rebuild and test its dependencies and self.

\end{itemize}

\subsection{Can use CFASLs to capture COMPILE-TIME side-effects}
\begin{itemize}
\item
Vast speed improvement, fewer rebuilds (the FASL may have changed yet the CFASL stay the same).

\item
Like C++ precompiled headers, except automatically deduced from the code.

\item
Was easily added to SBCL by Juho Snellman, could be as easily added to other compilers.

\item
Careful {\tt eval-when} discipline required (as with ASDF really, but now it is enforced).

\end{itemize}

\subsection{Automated migration path from ASDF}
\begin{itemize}
\item
XCVB accepts dependencies from XCVB systems to ASDF systems and vice-versa.

\item
Automatic migration of your ASDF system using Andreas Fuchs's {\tt asdf-dependency-grovel}.

\item
Compile-time Lisp state requires extending the dependency-detection tool.

\item
ASDF extensions will require according XCVB extensions.

\end{itemize}

%___________________________________________________________________________

\section{XCVB today}

Urgently needed.

\subsection{User friendliness}
\begin{itemize}
\item
Add documentation and examples.

\item
Better behavior in face of errors.

\end{itemize}

\subsection{More features}
\begin{itemize}
\item
Combine multiple projects, find them using a search path.

\item
Refine migration and compilation to deal with harder cases
(data files read at compile-time, computed lisp files, etc.).

\item
Have a more general model for staged builds
(multiple intermediate images, dynamic dependency computation).

\end{itemize}

\subsection{Actually migrate a critical mass of existing ASDF systems}
\begin{itemize}
\item
Support manual overrides when automation breaks down.

\item
Maintain until upstream adopts XCVB (if ever)
-- automated migration makes that possible.

\item
Provide a distribution system
(as in {\tt asdf-install}, {\tt mudballs} or {\tt clbuild}, etc.).

\item
Fully bootstrap XCVB (make ASDF optional).

\end{itemize}

\subsection{Refactor Internals}
\begin{itemize}
\item
Current implementation was a good first attempt, but needs to be reworked.

\item
Needs to be made more general to allow for desired and future features.

\item
Recognize hand-coded patterns, read literature, formalize a domain, grow a language.

\end{itemize}

%___________________________________________________________________________

\section{XCVB tomorrow}

The following improvements are enabled by
XCVB's deterministic separate compilation model.

\subsection{Distributed backends}
\begin{itemize}
\item
Pluggable distributed compilation ({\tt distcc} for CL).

\item
Take over the build, make it distributed with Erlang-in-Lisp.

\item
Requires compiler support to preserve source locations for debugging.

\end{itemize}

\subsection{Caching}
\begin{itemize}
\item
Cache objects rather than rebuild ({\tt ccache} for CL).

\item
Base cache on crypto hash fully capturing the computation and its dependencies.

\item
Can track all the modified dependencies
since last success at building and verifying a component.

\item
Push for more determinism in Lisp compilers!

\end{itemize}

\subsection{Dependency management}
\begin{itemize}
\item
{\tt xcvb-dependency-check} to detect superfluous dependencies
(to be based on {\tt asdf-dependency-grovel}).

\item
Cache above results to suggest missing dependencies.

\item
Actually implement dependency-based testing.

\item
Integrate test dependency tracking with code-coverage tools.

\end{itemize}

\subsection{Extend the build Specification Language}
\begin{itemize}
\item
Build rules that call arbitrary programs (as in a {\tt Makefile}).

\item
Computed source files, including from parametrized computations.

\item
Dependency on arbitrary computed features, only compiled once.

\item
Automated finalization and verification of modules.

\end{itemize}

\subsection{Manage reader extensions, alternate grammars, hygienic macros, etc.}
\begin{itemize}
\item
Made possible and convenient by separate compilation.

\item
No pollution of compile-time environment from other modules.

\item
Everyone can use whatever fits his purposes, with well-defined semantics.

\item
Requires compiler support to preserve source locations for debugging.

\end{itemize}

\subsection{Layer namespace management on top of it}
\begin{itemize}
\item
Automate evolution of {\tt defpackage} forms.

\item
More sensible replacement for packages (lexicons? modules as in PLT Scheme?).

\item
Higher-order parametric components (PLT Scheme units).

\item
Many levels of static typing with interface that enforces implicit contracts, etc.

\item
Generally, make CL competitive again wrt access to latest improvements from research.

\end{itemize}

\subsection{Abstract away the execution model}
\begin{itemize}
\item
Semantics: proper tail calls? continuations? serializable state? etc.

\item
Performance: debuggability? optimization levels?

\item
A file can require some of the above settings.

\item
A same module can be compiled according to many combinations of them.

\end{itemize}

%___________________________________________________________________________

\section{Need for extensions to the CL standard}

Short of reimplementing all of CL in a translation layer,
some of the above features cannot be implemented on top of standard CL:
they require access to functionality
below the standardized abstraction barrier of a CL implementation.
\subsection{Access to system functions}
\begin{itemize}
\item
open, fork, exec, sockets, etc. -- happily we have CFFI, IOLib.

\item
nothing specific to XCVB here, but still (sadly) deserves mentioning.

\end{itemize}

\subsection{Encapsulation of COMPILE-TIME side-effects}
\begin{itemize}
\item
CFASL only in SBCL for now.

\item
slow loading ``FAS''L can make do if you can cope with
intermixing LOAD-TIME side-effects.

\end{itemize}

\subsection{Encapsulation of LOAD-TIME partial state, not side-effects}
\begin{itemize}
\item
FASL is still too slow to load, cannot be shared between binaries.

\item
{\tt SB-HEAPDUMP} can be {\tt mmap}()ed -- but isn't even standard feature of SBCL.

\end{itemize}

\subsection{Programmable access to debugging meta-information}
\begin{itemize}
\item
Syntax extension requires support for recording source locations.

\item
Semantic layering is a challenge for single-stepping, access to high-level view of the state.

\item
Support multiple evaluation models in a given running environment.

\end{itemize}

\subsection{PCLSRing}

Generalizing PCLSRing \cite{PCLSRing},
when interrupting, inspecting or single-stepping a program in a higher-level language,
you don't want to handle intermediate states of the low-level implementation,
but safe points with a meaningful high-level interpretation.

\begin{itemize}
\item
Needed for transactionality in single-stepped and/or concurrent evaluation.

\item
Challenge: a good meta-level protocol for users to define PCLSRing for arbitrary semantic layers.

\item
With such a tool, all the system can be implemented with first-class translation layers.

\end{itemize}

%___________________________________________________________________________

\section{Conclusion}

%\subsection{Common Lisp is behind}

XCVB is nothing fancy -- just elaborate plumbing.
The ideas within are mostly well-known;
each but the most prospective of them
are already implemented in many build or module systems
for other languages.
Yet the bulk of the work is still ahead for XCVB.
That's how far behind
Common Lisp is with respect to modularity.

%\subsection{Social Rationale}

The deep rationale for XCVB is a social concern:
minimizing programmer-side cognitive burden in combining modules.
Technical and social aspects are tied in obvious ways,
yet most people wilfully ignore at least one of the two aspects.

%\subsection{Timeline}

XCVB was initially developed by Spencer Brody
during the Summer 2008 at ITA Software,
under the guidance of Fran�ois-Ren� Rideau.
Rideau briefly worked on it in mid December 2008
to release and document a usable prototype.

Our hope is that by the time the conference happens,
we will have already deployed XCVB on a large system,
and moved some points from ``XCVB today'' into ``XCVB yesterday''.
However as of this writing, this hasn't happened yet.

%\subsection{The Code}

XCVB code and documentation can be found at:
\center{\tt\href{http://common-lisp.net/projects/xcvb/}
		{http://common-lisp.net/projects/xcvb/}}

%\appendix
%\section{Appendix Title}

\acks
This presentation describes work done at ITA Software, Inc. \\
Many thanks to James Knight for the essential insights. \\
Thanks to Juho Snellman for CFASL support in SBCL.

\bibliography{xcvb}
\bibliographystyle{plainnat}

\end{document}

% 8.5"x 11" paper -- standard US Letter size.
% 4 pages max, 9pt font.
% http://www.sigplan.org/authorInformation.htm
% http://www.sigplan.org/sigplanconf.cls
% http://www.sigplan.org/sigplanconf-template.tex
